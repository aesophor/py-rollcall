\mySection{Experimental Results}

\subsection{Accuracy of Facial Recognition}
In our experiments, we select a variable n $\in$ $\{5, 10, 15, 20, 25, 30\}$ as our experimental parameter,
where n is the number of facial embeddings pre-computed per person. Simply put, n is the
number of photos collected from each individual before actually performing facial recognition.
Each time the experiment yields the number of faces recognized correctly, denoted by p.
We define another variable k to represent the number of faces in a testing set.
Finally, the correct rate $C_n$  for an experiment with respect to the parameter n can be evaluated by equation~\ref{eq:correct-rate}.

\begin{equation}
  \label{eq:correct-rate}
  C_n = p / k
\end{equation}

To assess how well our system can recognize faces with respect to different values of n,
we perform the experiment 6 times with varying values of n on each testing set. Moreover,
since we have three testing sets, we perform the experiment 18 times in total.
All values of $C_n$ collected from our experiments are presented in table~\ref{tab:exp-result-tab},
the visualization of which is shown in figure~\ref{fig:exp-result-chart}.
\vspace{0.5cm}

\begin{table}[!htb]
\centering
\caption{Correct rates for PyRollCall's facial recognition feature.} 
\begin{tabular}{@{}lcccccc@{}}
\toprule[2pt]
& \multicolumn{6}{c}{Number of embeddings pre-computed per person}                                                                                                                              \\ \addlinespace[0.5em]
                  & 5                          & 10                         & 15                         & 20                         & 25                         & 30                         \\ \midrule \addlinespace[0.5em]
Testing Set 1     & 8/20                       & 10/20                      & 15/20                      & 17/20                      & 17/20                      & 17/20                      \\
                  & \multicolumn{1}{c}{(40\%)} & \multicolumn{1}{c}{(50\%)} & \multicolumn{1}{c}{(75\%)} & \multicolumn{1}{c}{(85\%)} & \multicolumn{1}{c}{(85\%)} & \multicolumn{1}{c}{(85\%)} \\ \addlinespace[0.5em] \midrule \addlinespace[0.5em]
Testing Set 2     & 8/20                       & 10/20                      & 12/20                      & 13/20                      & 14/20                      & 14/20                      \\
                  & \multicolumn{1}{c}{(40\%)} & \multicolumn{1}{c}{(50\%)} & \multicolumn{1}{c}{(60\%)} & \multicolumn{1}{c}{(65\%)} & \multicolumn{1}{c}{(70\%)} & \multicolumn{1}{c}{(70\%)} \\ \addlinespace[0.5em] \midrule \addlinespace[0.5em]
Testing Set 3     & 9/20                       & 13/20                      & 14/20                      & 15/20                      & 15/20                      & 15/20                      \\
                  & \multicolumn{1}{c}{(45\%)} & \multicolumn{1}{c}{(65\%)} & \multicolumn{1}{c}{(70\%)} & \multicolumn{1}{c}{(75\%)} & \multicolumn{1}{c}{(75\%)} & \multicolumn{1}{c}{(75\%)} \\ \addlinespace[0.5em] \midrule[2pt] \addlinespace[0.5em]
Avg. Correct Rate & 41.67\%                    & 55\%                       & 68.33\%                    & 75\%                       & 76.67\%                    & 76.67\%                    \\ \addlinespace[0.5em]
\bottomrule[2pt]
\end{tabular}
\label{tab:exp-result-tab}
\end{table}

The results indicate that PyRollCall can achieve approximately \textbf{70\% to 85\%} of correct rate
for facial recognition at its best if \textbf{25 to 30} photos are collected in advance from each individual.
\vspace{0.5cm}

\begin{figure}[!htb]
  \centering
  \includegraphics[width=0.8\linewidth]{figures/exp-result-chart.png}
  \caption{Visualization of the correct rates for PyRollCall's facial recognition feature.}
  \label{fig:exp-result-chart}
\end{figure}



\subsection{Analysis of Successful Facial Recognition}
In facial recognition, various factors such as face occlusion, face tilt, head rotation and aging
can pose a great challenge, and since our system is aimed to be practically used by educational
organization, we have taken these factors into consideration.
For instance, we have the legendary Hong Kong actor, Star Chow, as one of our participants. His photos
in the training set covers various facial expressions, hairstyles, angle of faces and even different age.
Figure~\ref{fig:correct-recog} demonstrates that even though most of his photos in the training set
are of the young Star Chow without wearing glasses, PyRollCall can still successfully recognize
the mid-aged Star Chow with glasses put on.

\begin{figure}[!htb]
  \centering
  \begin{subfigure}[b]{0.65\linewidth}
    \includegraphics[width=\linewidth]{figures/star-chow-training-set.png}
    \caption{Star Chow's Training Set}
  \end{subfigure}
  \hfill
  \begin{subfigure}[b]{0.3\linewidth}
    \includegraphics[width=\linewidth]{figures/star-chow-success.png}
    \caption{Star Chow}
  \end{subfigure}
  \caption{Example of successful facial recognition.}
  \label{fig:correct-recog}
\end{figure}
\vspace{0.5cm}



\subsection{Analysis of Faulty Facial Recognition}
In most cases, PyRollCall can recognize faces correctly as long as 25 to 30 photos are collected
beforehand from each individual. However, sometimes erroneous facial recognition can still take place,
especially when the targets wear makeup. Figure~\ref{fig:false-recog1} shows an example where
faulty facial recognition occurs due to makeup, rendering the essential structures of two faces
similar to each other.
\vspace{0.2cm}

\begin{figure}[!htb]
  \centering
  \begin{subfigure}[b]{0.3\linewidth}
    \includegraphics[width=\linewidth]{figures/false-recog-correct1.png}
    \caption{Ke Jia Yan}
  \end{subfigure}
  \begin{subfigure}[b]{0.3\linewidth}
    \includegraphics[width=\linewidth]{figures/false-recog-error1.png}
    \caption{Wang Li Kun}
  \end{subfigure}
  \caption{Erroneous facial recognition, example 1.}
  \label{fig:false-recog1}
\end{figure}

Another reason which can lead to error in facial recognition is insufficient amount of photos collected,
as shown in figure~\ref{fig:false-recog2}. Even with 30 photos collected for each individual in advance,
sometimes the result can still be faulty.% Collecting 30 photos from each individual is already pretty
%difficult, it would be impractical to require students to provide more than 30 photos.
\vspace{0.2cm}

\begin{figure}[!htb]
  \centering
  \begin{subfigure}[b]{0.3\linewidth}
    \includegraphics[width=\linewidth]{figures/false-recog-correct2.png}
    \caption{Jay Chou}
  \end{subfigure}
  \begin{subfigure}[b]{0.3\linewidth}
    \includegraphics[width=\linewidth]{figures/false-recog-error2.png}
    \caption{Jam Hsiao}
  \end{subfigure}
  \caption{Erroneous facial recognition, example 2.}
  \label{fig:false-recog2}
\end{figure}
\vspace{0.2cm}



\subsection{Experiment on Educational Organization}
Finally, in this section we discuss the result of another experiment performed
in real world scenarios. In this experiment, we have 30 students as our participants,
and we collected approximately 70 photos from each student before carrying out the experiment
which cover various angles of each face.~\ref{fig:precollect-example} shows the example of photos
pre-collected from a student.

\begin{figure}[!htb]
  \centering
  \begin{subfigure}[b]{0.18\linewidth}
    \includegraphics[height=3cm,keepaspectratio]{figures/me/upper-right.png}
    \caption{Upper Right}
  \end{subfigure}
  \begin{subfigure}[b]{0.18\linewidth}
    \includegraphics[height=3cm,keepaspectratio]{figures/me/upper-right2.png}
    \caption{Upper Right2}
  \end{subfigure}
  \begin{subfigure}[b]{0.18\linewidth}
    \includegraphics[height=3cm,keepaspectratio]{figures/me/upper-front.png}
    \caption{Upper Front}
  \end{subfigure}
  \begin{subfigure}[b]{0.18\linewidth}
    \includegraphics[height=3cm,keepaspectratio]{figures/me/upper-left2.png}
    \caption{Upper Left 2}
  \end{subfigure}
  \begin{subfigure}[b]{0.18\linewidth}
    \includegraphics[height=3cm,keepaspectratio]{figures/me/upper-left.png}
    \caption{Upper Left}
  \end{subfigure}

  \begin{subfigure}[b]{0.18\linewidth}
    \includegraphics[height=3cm,keepaspectratio]{figures/me/mid-right.png}
    \caption{Mid Right}
  \end{subfigure}
  \begin{subfigure}[b]{0.18\linewidth}
    \includegraphics[height=3cm,keepaspectratio]{figures/me/mid-right2.png}
    \caption{Mid Right 2}
  \end{subfigure}
  \begin{subfigure}[b]{0.18\linewidth}
    \includegraphics[height=3cm,keepaspectratio]{figures/me/mid-front.png}
    \caption{Mid Front}
  \end{subfigure}
  \begin{subfigure}[b]{0.18\linewidth}
    \includegraphics[height=3cm,keepaspectratio]{figures/me/mid-left2.png}
    \caption{Mid Left 2}
  \end{subfigure}
  \begin{subfigure}[b]{0.18\linewidth}
    \includegraphics[height=3cm,keepaspectratio]{figures/me/mid-left.png}
    \caption{Mid Left}
  \end{subfigure}

   \begin{subfigure}[b]{0.18\linewidth}
    \includegraphics[height=3cm,keepaspectratio]{figures/me/lower-right.png}
    \caption{Lower Right}
  \end{subfigure}
  \begin{subfigure}[b]{0.18\linewidth}
    \includegraphics[height=3cm,keepaspectratio]{figures/me/lower-right2.png}
    \caption{Lower Right 2}
  \end{subfigure}
  \begin{subfigure}[b]{0.18\linewidth}
    \includegraphics[height=3cm,keepaspectratio]{figures/me/lower-front.png}
    \caption{Lower Front}
  \end{subfigure}
  \begin{subfigure}[b]{0.18\linewidth}
    \includegraphics[height=3cm,keepaspectratio]{figures/me/lower-left2.png}
    \caption{Lower Left 2}
  \end{subfigure}
  \begin{subfigure}[b]{0.18\linewidth}
    \includegraphics[height=3cm,keepaspectratio]{figures/me/lower-left.png}
    \caption{Lower Left}
  \end{subfigure}
  \caption{Example of pre-collected photos from a participant.}
  \label{fig:precollect-example}
\end{figure}
\vspace{0.5cm}

However, even with more photos pre-collected from each student,
this real-world experiment doesn't achieve the same accuracy for facial recognition
as the previous experiment did. The experimental results are demonstrated in figure~\ref{fig:utaipei-exp-result},
and the face detection rates and facial recognition accuracies are presented in table~\ref{tab:utaipei-exp-result}.
\vspace{0.5cm}

\begin{figure}[!htb]
  \centering
  \includegraphics[width=\linewidth]{figures/utaipei/1.png}
  \caption{Experimental result 1.}
  \label{fig:utaipei-exp-result}
\end{figure}
\newpage

\begin{figure}[!htb]
  \centering
  \includegraphics[width=\linewidth]{figures/utaipei/2.png}
  \caption{Experimental result 2.}
  \label{fig:utaipei-exp-result}
\end{figure}
\vspace{0.5cm}

\begin{figure}[!htb]
  \centering
  \includegraphics[width=\linewidth]{figures/utaipei/3.png}
  \caption{Experimental result 3.}
  \label{fig:utaipei-exp-result}
\end{figure}
\newpage

\begin{figure}[!htb]
  \centering
  \includegraphics[width=\linewidth]{figures/utaipei/4.png}
  \caption{Experimental result 4.}
  \label{fig:utaipei-exp-result}
\end{figure}
\vspace{0.5cm}

\begin{figure}[!htb]
  \centering
  \includegraphics[width=\linewidth]{figures/utaipei/5.png}
  \caption{Experimental result 5.}
  \label{fig:utaipei-exp-result}
\end{figure}
\newpage


\begin{table}[]
\centering
\caption{Experimental Results in University of Taipei.}
\begin{tabular}{@{}lcccccc@{}}
\toprule[2pt]
Experiment & Face Detection Rate & Facial Recognition Accuracy \\ \midrule \addlinespace[0.5em]
1          & 10/15 (67\%)        & 4/10 (40\%)                 \\ \midrule \addlinespace[0.5em]
2          & 12/13 (92\%)        & 3/12 (25\%)                 \\ \midrule \addlinespace[0.5em]
3          & 11/12 (92\%)        & 4/11 (36\%)                 \\ \midrule \addlinespace[0.5em]
4          & 16/18 (89\%)        & 2/16 (13\%)                 \\ \midrule \addlinespace[0.5em]
5          & 11/14 (79\%)        & 3/11 (27\%)                 \\ \addlinespace[0.5em] \midrule[2pt] \addlinespace[0.5em]
Avg        & 83.8\%              & 28.2\%                      \\
\bottomrule[2pt]
\end{tabular}
\label{tab:utaipei-exp-result}
\end{table}

\vspace{0.5cm}
By observing the experimental result, as well as comparing the photos pre-collected
in the previous experiment with the photos pre-collected in this experiment,
we conclude several factors that could lead to erroneous facial recognition results:
\vspace{0.5cm}

\setstretch{1.0}
\begin{itemize}
  \item The faces of the male celebrities used in the previous experiment are mostly distinguishable between one another,
    thus achieving very good results.
  \item The faces of the female celebrities used in the previous experiment are distinguishable by human eyes, but apparently
    this is not the case for computers since the error rate for them are quite high.
  \item Certain individual with highly distinguishable facial features has almost 100\% correct rate for facial recognition,
    but for most students the correct rates are relatively low.
  \item Studies \cite{err-facial-recog-tech} have shown that facial recognition technologies can produce widely inaccurate results, especially for non-whites.
  \end{itemize}
\setstretch{\myContentLineSpacing}


Although the accuracy of facial recognition is not ideal enough, the system architecture proposed in this thesis
still remains valuable and the facial recognition module (pyrollcall/face.py) can be replaced with better approaches
in future researches.
\vspace{0.2cm}

