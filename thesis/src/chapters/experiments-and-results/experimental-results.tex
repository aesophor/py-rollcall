\mySection{Experimental Results}
We want to find out how many photos must be collected from each individual in order to have PyRollCall reach its best performance.
The results indicate that PyRollCall can achieve approximately \textbf{70\% to 85\%} of correct rate for facial recognition at its best
if \textbf{25 to 30} photos are collected in advance from each individual.
The raw data of our experiments is presented in table~\ref{tab:exp-result-tab},
and we also visualize the data with a line chart, as shown in figure~\ref{fig:exp-result-chart}.
\vspace{0.3cm}

\begin{table}[!htb]
\centering
\caption{Correct rate for PyRollCall's facial recognition feature.} 
\begin{tabular}{@{}lcccccc@{}}
\toprule[2pt]
& \multicolumn{6}{c}{Number of embeddings pre-computed per person}                                                                                                               \\ \addlinespace[0.5em]
              & 5                          & 10                         & 15                         & 20                         & 25                         & 30                         \\ \midrule \addlinespace[0.5em]
Testing Set 1 & 8/20                       & 10/20                      & 15/20                      & 17/20                      & 17/20                      & 17/20                      \\
              & \multicolumn{1}{c}{(40\%)} & \multicolumn{1}{c}{(50\%)} & \multicolumn{1}{c}{(75\%)} & \multicolumn{1}{c}{(85\%)} & \multicolumn{1}{c}{(85\%)} & \multicolumn{1}{c}{(85\%)} \\ \addlinespace[0.5em] \midrule \addlinespace[0.5em]
Testing Set 2 & 8/20                       & 10/20                      & 12/20                      & 13/20                      & 14/20                      & 14/20                      \\
              & \multicolumn{1}{c}{(40\%)} & \multicolumn{1}{c}{(50\%)} & \multicolumn{1}{c}{(60\%)} & \multicolumn{1}{c}{(65\%)} & \multicolumn{1}{c}{(70\%)} & \multicolumn{1}{c}{(70\%)} \\ \addlinespace[0.5em] \midrule \addlinespace[0.5em]
Testing Set 3 & 9/20                       & 13/20                      & 14/20                      & 15/20                      & 15/20                      & 15/20                      \\
              & \multicolumn{1}{c}{(45\%)} & \multicolumn{1}{c}{(65\%)} & \multicolumn{1}{c}{(70\%)} & \multicolumn{1}{c}{(75\%)} & \multicolumn{1}{c}{(75\%)} & \multicolumn{1}{c}{(75\%)} \\ \addlinespace[0.5em]
\bottomrule[2pt]
\end{tabular}
\label{tab:exp-result-tab}
\end{table}
\vspace{0.2cm}

\begin{figure}[!htb]
  \centering
  \includegraphics[width=0.85\linewidth]{figures/exp-result-chart.png}
  \caption{Visualization of the correct rate for PyRollCall's facial recognition feature.}
  \label{fig:exp-result-chart}
\end{figure}
\clearpage


In most cases, PyRollCall can recognize faces correctly as long as 25 to 30 photos are collected
beforehand from each individual. However, sometimes erroneous facial recognition can still take place,
especially if the targets are wearing makeup. Figure~\ref{fig:false-recog1} shows an example where
faulty facial recognition happens due to makeup.
\vspace{0.2cm}

\begin{figure}[!htb]
  \centering
  \begin{subfigure}[b]{0.3\linewidth}
    \includegraphics[width=\linewidth]{figures/false-recog-correct1.png}
    \caption{Ke Jia Yan}
  \end{subfigure}
  \begin{subfigure}[b]{0.3\linewidth}
    \includegraphics[width=\linewidth]{figures/false-recog-error1.png}
    \caption{Wang Li Kun}
  \end{subfigure}
  \caption{Erroneous facial recognition, example 1.}
  \label{fig:false-recog1}
\end{figure}


Another reason which can lead to error in facial recognition is insufficient amount of photos collected,
as shown in figure~\ref{fig:false-recog2}. Even with 30 photos collected for each individual in advance,
sometimes the result can still be faulty.
\vspace{0.2cm}

\begin{figure}[!htb]
  \centering
  \begin{subfigure}[b]{0.3\linewidth}
    \includegraphics[width=\linewidth]{figures/false-recog-correct2.png}
    \caption{Jay Chou}
  \end{subfigure}
  \begin{subfigure}[b]{0.3\linewidth}
    \includegraphics[width=\linewidth]{figures/false-recog-error2.png}
    \caption{Jam Hsiao}
  \end{subfigure}
  \caption{Erroneous facial recognition, example 2.}
  \label{fig:false-recog2}
\end{figure}
\vspace{0.5cm}


Next, we
With sufficient photos of students provided and their facial embeddings pre-computed,
the system will be able to detect and recognize faces correctly. Currently, as shown

to save time in classes, teachers and students will have to spend equivilently extra amount of time
before classes on the tasks such as collecting photos and pre-computing facial embeddings.
This proves that there is a trade off between convenience and efficiency.
\vspace{0.2cm}

