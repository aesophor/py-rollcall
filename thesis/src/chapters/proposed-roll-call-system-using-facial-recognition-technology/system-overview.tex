\mySection{System Overview}
PyRollCall is able to run on Windows, macOS and unix-like systems as long as
the system has Python3 installed. All of its external dependencies are already packaged
in the project via Python's \emph{virtual environment}, making its installation easier.

\vspace*{0.2cm}

\begin{lstlisting}[numbers=none,xleftmargin=0em]
-rw-r--r-- 1 aesophor aesophor 1.2K Sep 26 20:33 course.py
-rw-r--r-- 1 aesophor aesophor 3.1K Sep 26 20:33 database.py
-rw-r--r-- 1 aesophor aesophor 5.8K Jan 18 14:22 face.py
-rw-r--r-- 1 aesophor aesophor  493 Sep 26 20:33 __init__.py
-rw-r--r-- 1 aesophor aesophor  18K Sep 26 20:33 mainwindow.py
-rw-r--r-- 1 aesophor aesophor  399 Sep 26 20:33 pyrollcall.py
-rw-r--r-- 1 aesophor aesophor 2.0K Sep 26 20:33 session.py
-rw-r--r-- 1 aesophor aesophor  827 Sep 26 20:33 student.py
-rw-r--r-- 1 aesophor aesophor  399 Sep 26 20:33 utils.py
-rw-r--r-- 1 aesophor aesophor 4.2K Sep 26 20:33 widget.py
\end{lstlisting}

The system comes with an easy-to-use graphical user interface (GUI) crafted with PyGTK,
allowing teachers to easily
(1) Maintain the data of the course and students they teach,
(2) take photos of students via camera,
(3) train the network with the photos of students,
(4) perform roll calls and
(5) export students' attendance to files.

\begin{algorithm}  
\caption{A}  
\label{alg:A}  
\begin{algorithmic}  
\STATE {set $r(t)=x(t)$}   
\REPEAT   
\STATE set $h(t)=r(t)$   
\REPEAT  
\STATE set $h(t)=r(t)$   
\UNTIL{B}   
\UNTIL{B}  
\end{algorithmic}  
\end{algorithm}  
